% Options for packages loaded elsewhere
\PassOptionsToPackage{unicode}{hyperref}
\PassOptionsToPackage{hyphens}{url}
%
\documentclass[
]{article}
\usepackage{lmodern}
\usepackage{amsmath}
\usepackage{ifxetex,ifluatex}
\ifnum 0\ifxetex 1\fi\ifluatex 1\fi=0 % if pdftex
  \usepackage[T1]{fontenc}
  \usepackage[utf8]{inputenc}
  \usepackage{textcomp} % provide euro and other symbols
  \usepackage{amssymb}
\else % if luatex or xetex
  \usepackage{unicode-math}
  \defaultfontfeatures{Scale=MatchLowercase}
  \defaultfontfeatures[\rmfamily]{Ligatures=TeX,Scale=1}
\fi
% Use upquote if available, for straight quotes in verbatim environments
\IfFileExists{upquote.sty}{\usepackage{upquote}}{}
\IfFileExists{microtype.sty}{% use microtype if available
  \usepackage[]{microtype}
  \UseMicrotypeSet[protrusion]{basicmath} % disable protrusion for tt fonts
}{}
\makeatletter
\@ifundefined{KOMAClassName}{% if non-KOMA class
  \IfFileExists{parskip.sty}{%
    \usepackage{parskip}
  }{% else
    \setlength{\parindent}{0pt}
    \setlength{\parskip}{6pt plus 2pt minus 1pt}}
}{% if KOMA class
  \KOMAoptions{parskip=half}}
\makeatother
\usepackage{xcolor}
\IfFileExists{xurl.sty}{\usepackage{xurl}}{} % add URL line breaks if available
\IfFileExists{bookmark.sty}{\usepackage{bookmark}}{\usepackage{hyperref}}
\hypersetup{
  hidelinks,
  pdfcreator={LaTeX via pandoc}}
\urlstyle{same} % disable monospaced font for URLs
\usepackage[margin=1in]{geometry}
\usepackage{color}
\usepackage{fancyvrb}
\newcommand{\VerbBar}{|}
\newcommand{\VERB}{\Verb[commandchars=\\\{\}]}
\DefineVerbatimEnvironment{Highlighting}{Verbatim}{commandchars=\\\{\}}
% Add ',fontsize=\small' for more characters per line
\usepackage{framed}
\definecolor{shadecolor}{RGB}{248,248,248}
\newenvironment{Shaded}{\begin{snugshade}}{\end{snugshade}}
\newcommand{\AlertTok}[1]{\textcolor[rgb]{0.94,0.16,0.16}{#1}}
\newcommand{\AnnotationTok}[1]{\textcolor[rgb]{0.56,0.35,0.01}{\textbf{\textit{#1}}}}
\newcommand{\AttributeTok}[1]{\textcolor[rgb]{0.77,0.63,0.00}{#1}}
\newcommand{\BaseNTok}[1]{\textcolor[rgb]{0.00,0.00,0.81}{#1}}
\newcommand{\BuiltInTok}[1]{#1}
\newcommand{\CharTok}[1]{\textcolor[rgb]{0.31,0.60,0.02}{#1}}
\newcommand{\CommentTok}[1]{\textcolor[rgb]{0.56,0.35,0.01}{\textit{#1}}}
\newcommand{\CommentVarTok}[1]{\textcolor[rgb]{0.56,0.35,0.01}{\textbf{\textit{#1}}}}
\newcommand{\ConstantTok}[1]{\textcolor[rgb]{0.00,0.00,0.00}{#1}}
\newcommand{\ControlFlowTok}[1]{\textcolor[rgb]{0.13,0.29,0.53}{\textbf{#1}}}
\newcommand{\DataTypeTok}[1]{\textcolor[rgb]{0.13,0.29,0.53}{#1}}
\newcommand{\DecValTok}[1]{\textcolor[rgb]{0.00,0.00,0.81}{#1}}
\newcommand{\DocumentationTok}[1]{\textcolor[rgb]{0.56,0.35,0.01}{\textbf{\textit{#1}}}}
\newcommand{\ErrorTok}[1]{\textcolor[rgb]{0.64,0.00,0.00}{\textbf{#1}}}
\newcommand{\ExtensionTok}[1]{#1}
\newcommand{\FloatTok}[1]{\textcolor[rgb]{0.00,0.00,0.81}{#1}}
\newcommand{\FunctionTok}[1]{\textcolor[rgb]{0.00,0.00,0.00}{#1}}
\newcommand{\ImportTok}[1]{#1}
\newcommand{\InformationTok}[1]{\textcolor[rgb]{0.56,0.35,0.01}{\textbf{\textit{#1}}}}
\newcommand{\KeywordTok}[1]{\textcolor[rgb]{0.13,0.29,0.53}{\textbf{#1}}}
\newcommand{\NormalTok}[1]{#1}
\newcommand{\OperatorTok}[1]{\textcolor[rgb]{0.81,0.36,0.00}{\textbf{#1}}}
\newcommand{\OtherTok}[1]{\textcolor[rgb]{0.56,0.35,0.01}{#1}}
\newcommand{\PreprocessorTok}[1]{\textcolor[rgb]{0.56,0.35,0.01}{\textit{#1}}}
\newcommand{\RegionMarkerTok}[1]{#1}
\newcommand{\SpecialCharTok}[1]{\textcolor[rgb]{0.00,0.00,0.00}{#1}}
\newcommand{\SpecialStringTok}[1]{\textcolor[rgb]{0.31,0.60,0.02}{#1}}
\newcommand{\StringTok}[1]{\textcolor[rgb]{0.31,0.60,0.02}{#1}}
\newcommand{\VariableTok}[1]{\textcolor[rgb]{0.00,0.00,0.00}{#1}}
\newcommand{\VerbatimStringTok}[1]{\textcolor[rgb]{0.31,0.60,0.02}{#1}}
\newcommand{\WarningTok}[1]{\textcolor[rgb]{0.56,0.35,0.01}{\textbf{\textit{#1}}}}
\usepackage{longtable,booktabs}
\usepackage{calc} % for calculating minipage widths
% Correct order of tables after \paragraph or \subparagraph
\usepackage{etoolbox}
\makeatletter
\patchcmd\longtable{\par}{\if@noskipsec\mbox{}\fi\par}{}{}
\makeatother
% Allow footnotes in longtable head/foot
\IfFileExists{footnotehyper.sty}{\usepackage{footnotehyper}}{\usepackage{footnote}}
\makesavenoteenv{longtable}
\usepackage{graphicx}
\makeatletter
\def\maxwidth{\ifdim\Gin@nat@width>\linewidth\linewidth\else\Gin@nat@width\fi}
\def\maxheight{\ifdim\Gin@nat@height>\textheight\textheight\else\Gin@nat@height\fi}
\makeatother
% Scale images if necessary, so that they will not overflow the page
% margins by default, and it is still possible to overwrite the defaults
% using explicit options in \includegraphics[width, height, ...]{}
\setkeys{Gin}{width=\maxwidth,height=\maxheight,keepaspectratio}
% Set default figure placement to htbp
\makeatletter
\def\fps@figure{htbp}
\makeatother
\setlength{\emergencystretch}{3em} % prevent overfull lines
\providecommand{\tightlist}{%
  \setlength{\itemsep}{0pt}\setlength{\parskip}{0pt}}
\setcounter{secnumdepth}{-\maxdimen} % remove section numbering
\ifluatex
  \usepackage{selnolig}  % disable illegal ligatures
\fi

\author{}
\date{\vspace{-2.5em}}

\begin{document}

\begin{longtable}[]{@{}l@{}}
\toprule
\endhead
title: ``Machine Learning Final Exam''\tabularnewline
author: ``Sumit Dutt Mishra''\tabularnewline
date: ``05/02/2021''\tabularnewline
output:\tabularnewline
word\_document: default\tabularnewline
html\_document: default\tabularnewline
pdf\_document: default\tabularnewline
\bottomrule
\end{longtable}

\#\#\#Business Situation \#\#\#\#CRISA is an Asian market research
agency that specializes in tracking consumer purchase behavior in
consumer goods (both durable and nondurable). In one major research
project, CRISA tracks numerous consumer product categories (e.g.,
``detergents''), and, within each category, perhaps dozens of brands. To
track purchase behavior, CRISA constituted household panels in over 100
cities and towns in India, covering most of the Indian urban market. The
households were carefully selected using stratified sampling to ensure a
representative sample; a subset of 600 records is analyzed here. The
strata were defined on the basis of socioeconomic status and the market
(a collection of cities). CRISA has both transaction data (each row is a
transaction) and household data (each row is a household), and for the
household data it maintains the following information: \#\#\#\#\#●
Demographics of the households (updated annually) \#\#\#\#\#● Possession
of durable goods (car, washing machine, etc., updated annually; an
``affluence index'' is computed from this information) \#\#\#\#\#●
Purchase data of product categories and brands (updated monthly)
\#\#\#\#CRISA has two categories of clients: (1) advertising agencies
that subscribe to the database services, obtain updated data every
month, and use the data to advise their clients on advertising and
promotion strategies; (2) consumer goods manufacturers, which monitor
their market share using the CRISA database.

\begin{Shaded}
\begin{Highlighting}[]
\FunctionTok{library}\NormalTok{(dplyr)}
\end{Highlighting}
\end{Shaded}

\begin{verbatim}
## 
## Attaching package: 'dplyr'
\end{verbatim}

\begin{verbatim}
## The following objects are masked from 'package:stats':
## 
##     filter, lag
\end{verbatim}

\begin{verbatim}
## The following objects are masked from 'package:base':
## 
##     intersect, setdiff, setequal, union
\end{verbatim}

\begin{Shaded}
\begin{Highlighting}[]
\FunctionTok{library}\NormalTok{(ISLR)}
\FunctionTok{library}\NormalTok{(caret)}
\end{Highlighting}
\end{Shaded}

\begin{verbatim}
## Loading required package: lattice
\end{verbatim}

\begin{verbatim}
## Loading required package: ggplot2
\end{verbatim}

\begin{Shaded}
\begin{Highlighting}[]
\FunctionTok{library}\NormalTok{(factoextra)}
\end{Highlighting}
\end{Shaded}

\begin{verbatim}
## Welcome! Want to learn more? See two factoextra-related books at https://goo.gl/ve3WBa
\end{verbatim}

\begin{Shaded}
\begin{Highlighting}[]
\FunctionTok{library}\NormalTok{(GGally)}
\end{Highlighting}
\end{Shaded}

\begin{verbatim}
## Registered S3 method overwritten by 'GGally':
##   method from   
##   +.gg   ggplot2
\end{verbatim}

\begin{Shaded}
\begin{Highlighting}[]
\FunctionTok{set.seed}\NormalTok{(}\DecValTok{123}\NormalTok{)}
\end{Highlighting}
\end{Shaded}

\#\#\#Using k-means clustering for identifying clusters of households
based on: \#\#\#\#a. The variables that describe purchase behavior
(including brand loyalty) \#\#\#\#b. The variables that describe the
basis for purchase \#\#\#\#c.~The variables that describe both purchase
behavior and basis of purchase \#\#\#\#\#Reading And Cleaning the Data

\begin{Shaded}
\begin{Highlighting}[]
\NormalTok{BathSoap }\OtherTok{\textless{}{-}} \FunctionTok{read.csv}\NormalTok{(}\StringTok{"\textasciitilde{}/Downloads/BathSoap.csv"}\NormalTok{)}
\NormalTok{BathsoapData }\OtherTok{\textless{}{-}} \FunctionTok{data.frame}\NormalTok{(}\FunctionTok{sapply}\NormalTok{(BathSoap, }\ControlFlowTok{function}\NormalTok{(x) }\FunctionTok{as.numeric}\NormalTok{(}\FunctionTok{gsub}\NormalTok{(}\StringTok{"\%"}\NormalTok{, }\StringTok{""}\NormalTok{, x))))}
\end{Highlighting}
\end{Shaded}

\#\#\#\#We gathered data from branded purchases based on the customer's
purchase percentage on the Brand code, then found the highest brand
loyal percentage and compared it to the other 999 brand purchases to
determine brand loyalty.

\#\#\#\#The Max Brand purchase percentage is higher than the Other Brand
purchase percentage when a customer is loyal to a company. As a result,
the customer develops a sense of brand loyalty.

\#\#\#\#We are using k-means clustering to group attributes into 2
sections:(i) Loyal Customers to Brand (ii) Disloyal Customers to Brand .
For this we are taking k=2

\begin{Shaded}
\begin{Highlighting}[]
\NormalTok{Loyalcustomers }\OtherTok{\textless{}{-}}\NormalTok{ BathsoapData[,}\DecValTok{23}\SpecialCharTok{:}\DecValTok{31}\NormalTok{]}
\NormalTok{Loyalcustomers}\SpecialCharTok{$}\NormalTok{MaxBrand }\OtherTok{\textless{}{-}} \FunctionTok{apply}\NormalTok{(Loyalcustomers,}\DecValTok{1}\NormalTok{,max)}
\NormalTok{BathSoapBrandLoyalty }\OtherTok{\textless{}{-}} \FunctionTok{cbind}\NormalTok{(BathsoapData[,}\FunctionTok{c}\NormalTok{(}\DecValTok{19}\NormalTok{, }\DecValTok{13}\NormalTok{, }\DecValTok{15}\NormalTok{, }\DecValTok{12}\NormalTok{, }\DecValTok{31}\NormalTok{, }\DecValTok{14}\NormalTok{, }\DecValTok{16}\NormalTok{,}\DecValTok{20}\NormalTok{)], }\AttributeTok{MaximumLoyal =}\NormalTok{ Loyalcustomers}\SpecialCharTok{$}\NormalTok{MaxBrand)}
\NormalTok{BathSoapBrandLoyalty }\OtherTok{\textless{}{-}} \FunctionTok{scale}\NormalTok{(BathSoapBrandLoyalty)}
\NormalTok{Kmeans\_model\_1 }\OtherTok{\textless{}{-}} \FunctionTok{kmeans}\NormalTok{(BathSoapBrandLoyalty, }\AttributeTok{centers =} \DecValTok{2}\NormalTok{, }\AttributeTok{nstart =} \DecValTok{25}\NormalTok{)}
\NormalTok{BathSoapBrandLoyalty }\OtherTok{\textless{}{-}} \FunctionTok{cbind}\NormalTok{(BathSoapBrandLoyalty, }\AttributeTok{Cluster =}\NormalTok{ Kmeans\_model\_1}\SpecialCharTok{$}\NormalTok{cluster)}
\FunctionTok{fviz\_cluster}\NormalTok{(Kmeans\_model\_1, }\AttributeTok{data =}\NormalTok{ BathSoapBrandLoyalty)}
\end{Highlighting}
\end{Shaded}

\includegraphics{FinalML_files/figure-latex/unnamed-chunk-3-1.pdf}
\#\#\#\#Customers in Cluster 1 are Loyal to Brand, while customers in
Cluster 2 are Disloyal to Brand because they are unconcerned regarding
the products.

\#\#\#\#If we even cluster them further and take k=4

\begin{Shaded}
\begin{Highlighting}[]
\NormalTok{Kmeans\_model\_1 }\OtherTok{\textless{}{-}} \FunctionTok{kmeans}\NormalTok{(BathSoapBrandLoyalty, }\AttributeTok{centers =} \DecValTok{4}\NormalTok{, }\AttributeTok{nstart =} \DecValTok{25}\NormalTok{)}
\NormalTok{BathSoapBrandLoyalty\_4 }\OtherTok{\textless{}{-}} \FunctionTok{cbind}\NormalTok{(BathSoapBrandLoyalty[,}\SpecialCharTok{{-}}\DecValTok{10}\NormalTok{], }\AttributeTok{Cluster =}\NormalTok{ Kmeans\_model\_1}\SpecialCharTok{$}\NormalTok{cluster)}
\FunctionTok{fviz\_cluster}\NormalTok{(Kmeans\_model\_1, }\AttributeTok{data =}\NormalTok{ BathSoapBrandLoyalty\_4)}
\end{Highlighting}
\end{Shaded}

\includegraphics{FinalML_files/figure-latex/unnamed-chunk-4-1.pdf}

\#\#\#\#\#Looking at the data for consumer purchase conduct.

\#\#\#\#We've taken into account all of the selling propositions, chosen
the best, and compared them to determine which are the most effective
selling propositions to consider.

\begin{Shaded}
\begin{Highlighting}[]
\NormalTok{BathSoap\_sellprep }\OtherTok{\textless{}{-}}\NormalTok{ BathsoapData[,}\DecValTok{36}\SpecialCharTok{:}\DecValTok{46}\NormalTok{]}
\NormalTok{BathSoap\_sellprep}\SpecialCharTok{$}\NormalTok{Max }\OtherTok{\textless{}{-}} \FunctionTok{apply}\NormalTok{(BathSoap\_sellprep,}\DecValTok{1}\NormalTok{,max)}
\NormalTok{BathSoap\_sellprep}\SpecialCharTok{$}\NormalTok{MaxBrand }\OtherTok{\textless{}{-}} \FunctionTok{colnames}\NormalTok{(BathSoap\_sellprep)[}\FunctionTok{apply}\NormalTok{(BathSoap\_sellprep,}\DecValTok{1}\NormalTok{,which.max)]}
\end{Highlighting}
\end{Shaded}

\#\#\#\#\#Categories that are close to the Price Catagories. The same
can be said for Promotions..

\begin{Shaded}
\begin{Highlighting}[]
\NormalTok{PriceCat }\OtherTok{\textless{}{-}}\NormalTok{ BathsoapData[,}\DecValTok{32}\SpecialCharTok{:}\DecValTok{35}\NormalTok{]}
\NormalTok{PriceCat}\SpecialCharTok{$}\NormalTok{Max }\OtherTok{\textless{}{-}} \FunctionTok{apply}\NormalTok{(PriceCat,}\DecValTok{1}\NormalTok{,max)}
\NormalTok{PriceCat}\SpecialCharTok{$}\NormalTok{MaxBrand }\OtherTok{\textless{}{-}} \FunctionTok{colnames}\NormalTok{(PriceCat)[}\FunctionTok{apply}\NormalTok{(PriceCat,}\DecValTok{1}\NormalTok{,which.max)]}
\FunctionTok{table}\NormalTok{(PriceCat}\SpecialCharTok{$}\NormalTok{MaxBrand)}
\end{Highlighting}
\end{Shaded}

\begin{verbatim}
## 
## Pr.Cat.1 Pr.Cat.2 Pr.Cat.3 Pr.Cat.4 
##      132      343       78       47
\end{verbatim}

\begin{Shaded}
\begin{Highlighting}[]
\NormalTok{Promotion }\OtherTok{\textless{}{-}}\NormalTok{ BathsoapData[,}\DecValTok{20}\SpecialCharTok{:}\DecValTok{22}\NormalTok{]}
\NormalTok{Promotion}\SpecialCharTok{$}\NormalTok{Max }\OtherTok{\textless{}{-}} \FunctionTok{apply}\NormalTok{(Promotion,}\DecValTok{1}\NormalTok{,max)}
\NormalTok{Promotion}\SpecialCharTok{$}\NormalTok{MaxBrand }\OtherTok{\textless{}{-}} \FunctionTok{colnames}\NormalTok{(Promotion)[}\FunctionTok{apply}\NormalTok{(Promotion,}\DecValTok{1}\NormalTok{,which.max)]}
\FunctionTok{table}\NormalTok{(Promotion}\SpecialCharTok{$}\NormalTok{MaxBrand)}
\end{Highlighting}
\end{Shaded}

\begin{verbatim}
## 
##  Pur.Vol.No.Promo.... Pur.Vol.Other.Promo..     Pur.Vol.Promo.6.. 
##                   595                     1                     4
\end{verbatim}

\#\#\#\#\#As a consequence, when assessing their effect, we've only
looked at the more effective Selling Propositions. \#\#\#\#\#Promotions
and price categories are in the same boat.

\begin{Shaded}
\begin{Highlighting}[]
\NormalTok{PurchBehaviour }\OtherTok{\textless{}{-}}\NormalTok{ BathsoapData[,}\FunctionTok{c}\NormalTok{(}\DecValTok{32}\NormalTok{,}\DecValTok{33}\NormalTok{,}\DecValTok{34}\NormalTok{,}\DecValTok{35}\NormalTok{,}\DecValTok{36}\NormalTok{,}\DecValTok{45}\NormalTok{)]}
\NormalTok{PurchBehaviour }\OtherTok{\textless{}{-}} \FunctionTok{scale}\NormalTok{(PurchBehaviour)}
\CommentTok{\#View(PurchBehaviour)}
\FunctionTok{fviz\_nbclust}\NormalTok{(PurchBehaviour, kmeans, }\AttributeTok{method =} \StringTok{"silhouette"}\NormalTok{)}
\end{Highlighting}
\end{Shaded}

\includegraphics{FinalML_files/figure-latex/unnamed-chunk-7-1.pdf}
\#\#\#\#To determine the customer's buying pattern, the K means
Clustering model is used. k = 4 will be used in this situation.

\begin{Shaded}
\begin{Highlighting}[]
\NormalTok{Purch\_model }\OtherTok{\textless{}{-}} \FunctionTok{kmeans}\NormalTok{(PurchBehaviour, }\AttributeTok{centers =} \DecValTok{4}\NormalTok{, }\AttributeTok{nstart =} \DecValTok{25}\NormalTok{)}
\NormalTok{PurchBehaviour }\OtherTok{\textless{}{-}} \FunctionTok{cbind}\NormalTok{(PurchBehaviour, }\AttributeTok{Cluster =}\NormalTok{ Purch\_model}\SpecialCharTok{$}\NormalTok{cluster)}
\CommentTok{\#View(PurchBehaviour)}
\FunctionTok{fviz\_cluster}\NormalTok{(Purch\_model, }\AttributeTok{data =}\NormalTok{ PurchBehaviour)}
\end{Highlighting}
\end{Shaded}

\includegraphics{FinalML_files/figure-latex/unnamed-chunk-8-1.pdf}
\#\#\#\#When creating a definition, we must now consider the customers'
brand loyalty as well as their buying behaviour.

\begin{Shaded}
\begin{Highlighting}[]
\NormalTok{LoyalPurch }\OtherTok{\textless{}{-}} \FunctionTok{cbind}\NormalTok{(BathSoapBrandLoyalty[,}\SpecialCharTok{{-}}\DecValTok{10}\NormalTok{], PurchBehaviour[,}\SpecialCharTok{{-}}\DecValTok{7}\NormalTok{])}
\FunctionTok{fviz\_nbclust}\NormalTok{(LoyalPurch, kmeans, }\AttributeTok{method =} \StringTok{"silhouette"}\NormalTok{)}
\end{Highlighting}
\end{Shaded}

\includegraphics{FinalML_files/figure-latex/unnamed-chunk-9-1.pdf}

\begin{Shaded}
\begin{Highlighting}[]
\NormalTok{KMeans\_All }\OtherTok{\textless{}{-}} \FunctionTok{kmeans}\NormalTok{(LoyalPurch, }\AttributeTok{centers =} \DecValTok{4}\NormalTok{, }\AttributeTok{nstart =} \DecValTok{25}\NormalTok{)}
\end{Highlighting}
\end{Shaded}

\#\#\#\#While plotting the model for k = 4 and k = 5, it is clearly
visible that the aspects can be resolved by using 4 clusters without
using 5. For this reason, we will use k=4 \#\#\#Selecting the best
segmentation and commenting on the characteristics and Developing a
model that classifies the data into these segments.

\begin{Shaded}
\begin{Highlighting}[]
\NormalTok{LoyalPurch}\OtherTok{\textless{}{-}} \FunctionTok{cbind}\NormalTok{(LoyalPurch, }\AttributeTok{Cluster =} \FunctionTok{as.data.frame}\NormalTok{(KMeans\_All}\SpecialCharTok{$}\NormalTok{cluster))}
\NormalTok{clusters }\OtherTok{\textless{}{-}} \FunctionTok{matrix}\NormalTok{(}\FunctionTok{c}\NormalTok{(}\StringTok{"1"}\NormalTok{,}\StringTok{"2"}\NormalTok{,}\StringTok{"3"}\NormalTok{,}\StringTok{"4"}\NormalTok{),}\AttributeTok{nrow =} \DecValTok{4}\NormalTok{)}
\NormalTok{LoyalPurch\_Centroid }\OtherTok{\textless{}{-}} \FunctionTok{cbind}\NormalTok{(clusters,}\FunctionTok{as.data.frame}\NormalTok{(KMeans\_All}\SpecialCharTok{$}\NormalTok{centers))}
\FunctionTok{ggparcoord}\NormalTok{(LoyalPurch\_Centroid,}
           \AttributeTok{columns =} \DecValTok{2}\SpecialCharTok{:}\DecValTok{16}\NormalTok{, }\AttributeTok{groupColumn =} \DecValTok{1}\NormalTok{,}
           \AttributeTok{showPoints =} \ConstantTok{TRUE}\NormalTok{, }
           \AttributeTok{title =} \StringTok{"Parallel Coordinate Plot for for Bathsoap Data {-} K = 4"}\NormalTok{,}
           \AttributeTok{alphaLines =} \FloatTok{0.5}\NormalTok{)}
\end{Highlighting}
\end{Shaded}

\includegraphics{FinalML_files/figure-latex/unnamed-chunk-10-1.pdf}

\#\#\#\#The Demographic result is computed for each cluster

\hypertarget{converting-the-demographic-values-of-each-cluster.}{%
\paragraph{Converting the demographic values of each
cluster.}\label{converting-the-demographic-values-of-each-cluster.}}

\begin{Shaded}
\begin{Highlighting}[]
\NormalTok{Demograph }\OtherTok{\textless{}{-}}\FunctionTok{cbind}\NormalTok{(BathsoapData[,}\DecValTok{2}\SpecialCharTok{:}\DecValTok{11}\NormalTok{], }\AttributeTok{ClusterVal =}\NormalTok{ KMeans\_All}\SpecialCharTok{$}\NormalTok{cluster)}
\NormalTok{Center\_1 }\OtherTok{\textless{}{-}} \FunctionTok{colMeans}\NormalTok{(Demograph[Demograph}\SpecialCharTok{$}\NormalTok{ClusterVal }\SpecialCharTok{==} \StringTok{"1"}\NormalTok{,])}
\NormalTok{Center\_2 }\OtherTok{\textless{}{-}} \FunctionTok{colMeans}\NormalTok{(Demograph[Demograph}\SpecialCharTok{$}\NormalTok{ClusterVal }\SpecialCharTok{==} \StringTok{"2"}\NormalTok{,])}
\NormalTok{Center\_3 }\OtherTok{\textless{}{-}} \FunctionTok{colMeans}\NormalTok{(Demograph[Demograph}\SpecialCharTok{$}\NormalTok{ClusterVal }\SpecialCharTok{==} \StringTok{"3"}\NormalTok{,])}
\NormalTok{Center\_4 }\OtherTok{\textless{}{-}} \FunctionTok{colMeans}\NormalTok{(Demograph[Demograph}\SpecialCharTok{$}\NormalTok{ClusterVal }\SpecialCharTok{==} \StringTok{"4"}\NormalTok{,])}
\NormalTok{Centroids }\OtherTok{\textless{}{-}} \FunctionTok{rbind}\NormalTok{(Center\_1, Center\_2, Center\_3, Center\_4)}
\FunctionTok{ggparcoord}\NormalTok{(Centroids,}
           \AttributeTok{columns =} \FunctionTok{c}\NormalTok{(}\DecValTok{1}\NormalTok{,}\DecValTok{5}\NormalTok{,}\DecValTok{6}\NormalTok{,}\DecValTok{7}\NormalTok{,}\DecValTok{8}\NormalTok{), }\AttributeTok{groupColumn =} \DecValTok{11}\NormalTok{,}
           \AttributeTok{showPoints =} \ConstantTok{TRUE}\NormalTok{, }
           \AttributeTok{title =} \StringTok{"Demographic Metrics for Bathsoap Data Plotted in Parallel Coordinate Plot{-} K = 4"}\NormalTok{,}
           \AttributeTok{alphaLines =} \FloatTok{0.5}\NormalTok{)}
\end{Highlighting}
\end{Shaded}

\includegraphics{FinalML_files/figure-latex/unnamed-chunk-11-1.pdf}

\#\#\#\#We are using a barplot because there are a few attributes that
are categorical.

\#\#\#\#Plotting Eating Habit Frequency (Not Specified,Vegetarian Who
Eats Eggs, Vegetarian,Non-Vegetarian):

\begin{Shaded}
\begin{Highlighting}[]
\FunctionTok{barplot}\NormalTok{(}\FunctionTok{table}\NormalTok{(BathsoapData}\SpecialCharTok{$}\NormalTok{FEH,KMeans\_All}\SpecialCharTok{$}\NormalTok{cluster), }\AttributeTok{xlab =} \StringTok{"Cluster"}\NormalTok{, }\AttributeTok{ylab =} \StringTok{"Frequency of Eating Habit"}\NormalTok{, }\AttributeTok{main =} \StringTok{"The Eating Habit Frequency for each cluster"}\NormalTok{)}
\end{Highlighting}
\end{Shaded}

\includegraphics{FinalML_files/figure-latex/unnamed-chunk-12-1.pdf}

\#\#\#\#Plotting the Frequency of Gender like Male, Female and NA:

\begin{Shaded}
\begin{Highlighting}[]
\FunctionTok{barplot}\NormalTok{(}\FunctionTok{table}\NormalTok{(BathsoapData}\SpecialCharTok{$}\NormalTok{SEX,KMeans\_All}\SpecialCharTok{$}\NormalTok{cluster), }\AttributeTok{xlab =} \StringTok{"Clusters"}\NormalTok{, }\AttributeTok{ylab =} \StringTok{"Frequency of Gender"}\NormalTok{, }\AttributeTok{main =} \StringTok{"The Gender Frequency for each cluster"}\NormalTok{)}
\end{Highlighting}
\end{Shaded}

\includegraphics{FinalML_files/figure-latex/unnamed-chunk-13-1.pdf}
\#\#\#\#The female population has a higher purchasing rate, as females
in clusters 1 and 2 have the most females.

\#\#\#\#Plotting the Television Availability Frequency as Availability,
Unavailability and Unspecified:

\begin{Shaded}
\begin{Highlighting}[]
\FunctionTok{barplot}\NormalTok{(}\FunctionTok{table}\NormalTok{(BathsoapData}\SpecialCharTok{$}\NormalTok{CS, KMeans\_All}\SpecialCharTok{$}\NormalTok{cluster), }\AttributeTok{xlab =} \StringTok{"Cluster"}\NormalTok{, }\AttributeTok{ylab =} \StringTok{"Frequency of Television availability"}\NormalTok{, }\AttributeTok{main =} \StringTok{"The frequency of television availability"}\NormalTok{)}
\end{Highlighting}
\end{Shaded}

\includegraphics{FinalML_files/figure-latex/unnamed-chunk-14-1.pdf}

\#\#\#\#Since almost everybody has access to television, a television
adverti

\end{document}
